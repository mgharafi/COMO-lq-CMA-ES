\textbf{f37} & \multicolumn{2}{@{}c@{}}{2.0 \quad} & \multicolumn{2}{@{}c@{}}{3956 \quad} & \multicolumn{2}{@{}c@{}}{1.5e5 \quad} & \multicolumn{2}{@{}c@{}}{2.5e6 \quad} & \multicolumn{2}{@{}c@{}}{4.7e6 \quad} & \multicolumn{2}{@{}c@{}}{1.1e7 \quad} & \multicolumn{2}{@{}c@{}}{1.1e7 \quad} & \multicolumn{2}{@{}c@{}}{1.1e7 \quad} & \multicolumn{2}{@{}c@{}}{1.2e7 \quad} & \multicolumn{2}{@{}c@{}}{1.2e7 \quad} & \multicolumn{2}{@{}c@{}}{1.2e7 \quad} & \multicolumn{2}{@{}c@{}}{2.7e7 \quad} & \multicolumn{2}{@{}c@{}}{2.7e7 \quad} & \multicolumn{2}{@{}c@{}|}{2.7e7} & 1 & /10\\\hline
\algAtables\hspace*{\fill} & \textbf{5} & \textbf{.0}\mbox{\tiny (9)} & \textbf{0} & \textbf{.62}\mbox{\tiny (0.4)} & \multicolumn{2}{@{\,}l@{\,}}{$\infty$} & \multicolumn{2}{@{\,}l@{\,}}{\textbf{$\infty$}} & \multicolumn{2}{@{\,}l@{\,}}{\textbf{$\infty$}} & \multicolumn{2}{@{\,}l@{\,}}{\textbf{$\infty$}} & \multicolumn{2}{@{\,}l@{\,}}{\textbf{$\infty$}} & \multicolumn{2}{@{\,}l@{\,}}{\textbf{$\infty$}} & \multicolumn{2}{@{\,}l@{\,}}{\textbf{$\infty$}} & \multicolumn{2}{@{\,}l@{\,}}{\textbf{$\infty$}} & \multicolumn{2}{@{\,}l@{\,}}{\textbf{$\infty$}} & \multicolumn{2}{@{\,}l@{\,}}{\textbf{$\infty$}} & \multicolumn{2}{@{\,}l@{\,}}{\textbf{$\infty$}} & \multicolumn{2}{@{\,}l@{\,}|}{$\infty$\,\textit{5e4}} & 0 & /10\\
\algBtables\hspace*{\fill} & 16 & \mbox{\tiny (35)} & 0 & .71\mbox{\tiny (0.5)} & \multicolumn{2}{@{\,}l@{\,}}{$\infty$} & \multicolumn{2}{@{\,}l@{\,}}{\textbf{$\infty$}} & \multicolumn{2}{@{\,}l@{\,}}{\textbf{$\infty$}} & \multicolumn{2}{@{\,}l@{\,}}{\textbf{$\infty$}} & \multicolumn{2}{@{\,}l@{\,}}{\textbf{$\infty$}} & \multicolumn{2}{@{\,}l@{\,}}{\textbf{$\infty$}} & \multicolumn{2}{@{\,}l@{\,}}{\textbf{$\infty$}} & \multicolumn{2}{@{\,}l@{\,}}{\textbf{$\infty$}} & \multicolumn{2}{@{\,}l@{\,}}{\textbf{$\infty$}} & \multicolumn{2}{@{\,}l@{\,}}{\textbf{$\infty$}} & \multicolumn{2}{@{\,}l@{\,}}{\textbf{$\infty$}} & \multicolumn{2}{@{\,}l@{\,}|}{$\infty$\,\textit{5e4}} & 0 & /10\\
\algCtables\hspace*{\fill} & 9 & .2\mbox{\tiny (15)} & 2 & .5\mbox{\tiny (2)} & \textbf{1} & \textbf{.6}\mbox{\tiny (2)} & \multicolumn{2}{@{\,}l@{\,}}{\textbf{$\infty$}} & \multicolumn{2}{@{\,}l@{\,}}{\textbf{$\infty$}} & \multicolumn{2}{@{\,}l@{\,}}{\textbf{$\infty$}} & \multicolumn{2}{@{\,}l@{\,}}{\textbf{$\infty$}} & \multicolumn{2}{@{\,}l@{\,}}{\textbf{$\infty$}} & \multicolumn{2}{@{\,}l@{\,}}{\textbf{$\infty$}} & \multicolumn{2}{@{\,}l@{\,}}{\textbf{$\infty$}} & \multicolumn{2}{@{\,}l@{\,}}{\textbf{$\infty$}} & \multicolumn{2}{@{\,}l@{\,}}{\textbf{$\infty$}} & \multicolumn{2}{@{\,}l@{\,}}{\textbf{$\infty$}} & \multicolumn{2}{@{\,}l@{\,}|}{$\infty$\,\textit{5e4}} & 0 & /10\\
\algDtables\hspace*{\fill} & 10 & \mbox{\tiny (16)} & 3 & .5\mbox{\tiny (3)} & 3 & .4\mbox{\tiny (4)} & \multicolumn{2}{@{\,}l@{\,}}{\textbf{$\infty$}} & \multicolumn{2}{@{\,}l@{\,}}{\textbf{$\infty$}} & \multicolumn{2}{@{\,}l@{\,}}{\textbf{$\infty$}} & \multicolumn{2}{@{\,}l@{\,}}{\textbf{$\infty$}} & \multicolumn{2}{@{\,}l@{\,}}{\textbf{$\infty$}} & \multicolumn{2}{@{\,}l@{\,}}{\textbf{$\infty$}} & \multicolumn{2}{@{\,}l@{\,}}{\textbf{$\infty$}} & \multicolumn{2}{@{\,}l@{\,}}{\textbf{$\infty$}} & \multicolumn{2}{@{\,}l@{\,}}{\textbf{$\infty$}} & \multicolumn{2}{@{\,}l@{\,}}{\textbf{$\infty$}} & \multicolumn{2}{@{\,}l@{\,}|}{$\infty$\,\textit{5e4}} & 0 & /10\\